\documentclass[10pt, a5paper]{scrartcl}
% Settings
\newcommand\problemset{5}
\newif\ifcomments
\commentsfalse % hide comments
%\commentstrue % show comments
\usepackage[english]{../../exercises}
\usepackage{amsmath,amssymb,enumerate,xcolor,eurosym,nicefrac}
\usepackage[margin=1.7cm]{geometry}

\begin{document}
\boardquestions


\begin{exercise}[Sufficient Statistics]
	You are given a data set $ x=x_{1}^{n} $ of $ n $ independent, geometrically distributed observations.
Show that $ \sum_{i=1}^{n} x_i $ is a sufficient statistic for the geometric distribution.
\end{exercise}


\begin{exercise}[Covariance]
	A coin is taken from a box containing three coins, which give heads with probability $p = \nicefrac{1}{3}, \nicefrac{1}{2}$, and $\nicefrac{2}{3}$. The mysterious coin is tossed 80 times, resulting in 49 heads and 31 tails.

	\begin{subex}
		What is the likelihood of this data for each type of coin and which coin gives the maximum likelihood?
	\end{subex}

	\begin{subex}
		Now suppose that we have a single coin with unknown probability $p$ of landing heads. Find the likelihood and log likelihood functions given the same data. What is the maximum likelihood estimate for $p$?
	\end{subex}
\end{exercise}


\begin{exercise}[Dice]
There are five fair dice each with a different number of sides: 4,6,8,12,20.
Jon picks one of them uniformly at random rolls it and reports a 13.

	\begin{subex}	
		Compute the posterior probability for each die to have generated this outcome.
	\end{subex}
	
	\begin{subex}	
		Compute the posterior probabilities if the result had been a 5 instead. \\
		\emph{Hint: Drawing a table may help here. And please do use a calculator!}
	\end{subex}
\end{exercise}



\begin{exercise}[Geometric maximum likelihood estimator]
	Recall that if $Y$ is  geometrically distributed, then $P(Y=y) = (1-\theta)^{y}\theta$.
	You are given a set of independent, geometrically distributed observations $x = x_{1}^{n}$.
	Find the maximum likelihood estimator of $\theta$.
\end{exercise}


\vfill
\credits{Adapted from MIT course 18-05 by Jeremy Orloff and Jonathan Bloom, see ocw.mit.edu}
\end{document}