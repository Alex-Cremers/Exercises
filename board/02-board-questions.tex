\documentclass[10pt, a5paper]{scrartcl}
% Settings
\newcommand\problemset{2}
\newif\ifcomments
\commentsfalse % hide comments
%\commentstrue % show comments
\usepackage[english]{../../exercises}
\usepackage{amsmath,amssymb,enumerate,xcolor}
\usepackage[margin=1.7cm]{geometry}

\begin{document}
\boardquestions

\begin{exercise}[Rolling Dice]
	\begin{subex}
		Every member of your group rolls his/her 20-sided die.	
	\end{subex}
	
	\begin{subex}
		Check if at least two of the die at your table show the same
	  outcome.
	\end{subex}
	
	\begin{subex}
		Repeat the experiment a few time and estimate the probability of
	  this event.	
	\end{subex}
	
	\begin{subex}
		Define the sample space and probability function for this
	  experiment.	
	\end{subex}
	
	\begin{subex}
		Compute the probability of the event $A$ that at least two of
	  the outcomes coincide. \emph{Hint: It might be easier to compute the probability
	  of the complement of $A$. What is this event?}
	\end{subex}
\end{exercise}

\begin{exercise}[Evil Squirrels]
Consider 1.000.000 squirrels, 100 of them are evil and the rest is nice. 
The proposed alarm goes of when presented with an evil squirrel 99$\%$ of the time. 
But, it also goes of 1$\%$ of the time when presented with a nice squirrel.

	\begin{subex}
		If a squirrel sets off the alarm, what's the probability that it is evil?
	\end{subex}
	
	\begin{subex}
		Based on this, should the evil squirrel detector be acquired?
	\end{subex}
\end{exercise}


\begin{exercise}[Monty Hall]
A candidate in a TV show has to pick one of 3 doors. Behind one of the doors there is a car, and there are goats behind each of the two other doors. After the candidate has chosen, one of the doors with a goat is opened. The candidate is then given the choice to switch doors or to stick with his original choice

	Based on probabilistic calculations, should the	candidate switch doors or stick with his initial choice after one of the goats has been revealed?
	\emph{Hint: Start off with drawing a tree and computing the probability of winning the car when always switching the doors.}
\end{exercise}


\vfill
\credits{Adapted from MIT course 18-05 by Jeremy Orloff and Jonathan Bloom, see ocw.mit.edu}
\end{document}