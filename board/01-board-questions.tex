\documentclass[10pt, a5paper]{scrartcl}
% Settings
\newcommand\problemset{1}
\newif\ifcomments
\commentsfalse % hide comments
%\commentstrue % show comments
\usepackage[english]{../../exercises}
\usepackage{amsmath,amssymb,enumerate,xcolor}
\usepackage[margin=1.7cm]{geometry}

\begin{document}
\boardquestions

\begin{exercise}[Olympics]
	There are 5 competitors in the 100m final. 
	In how many ways can gold, silver and bronze be awarded?
\end{exercise}


\begin{exercise}[Coin flips]
	\begin{subex}
		Count the number of ways to get exactly 3 heads in 10 coin flips.
	\end{subex}
	
	\begin{subex}
		For a fair coin, what is the probability of exactly 3 heads in 10 flips?
	\end{subex}
\end{exercise}


\begin{exercise}[Poker]
	A deck of cards contains 52 with \textbf{13 ranks} (2, 3, \ldots, 9, 10, J, Q, K, A) and \textbf{4 suits} ($ {\color{red} \heartsuit}, \spadesuit , {\color{red} \diamondsuit}, \clubsuit $).
	A poker hand is a set of \textbf{5 cards}. A \textbf{one-pair} is a poker hand with two cards having one rank and the remaining three cards having different ranks. 
	For example: $ \{2{\color{red} \heartsuit}, 2\spadesuit, 5{\color{red} \heartsuit}, 8\clubsuit, K{\color{red} \diamondsuit} \}  $

	\begin{subex}
		How many different 5-card hands have exactly one pair? \\
		\emph{Hint: practice with how many 2 card hands have exactly one pair.}\\
		\emph{Hint for hint: use the rule of product.}
	\end{subex}
	
	\begin{subex}	
		What is the probability of getting a one pair poker hand?
	\end{subex}
\end{exercise}



\vfill
\credits{Adapted from MIT course 18-05 by Jeremy Orloff and Jonathan Bloom, see ocw.mit.edu}
\end{document}