\documentclass{article}

\usepackage{amsmath}
\usepackage{amssymb}
\usepackage{calc}
\usepackage{fullpage}
\usepackage{hyperref}
\hypersetup{colorlinks=true, urlcolor=blue, breaklinks=true}

%\newcommand{\prop}[1]{\lbrack \! \lbrack #1 \rbrack \! \rbrack}

\newcommand{\philip}[1]{ \textcolor{red}{\textbf{Philip:} #1}}
\newcommand{\chris}[1]{ \textcolor{blue}{\textbf{Chris:} #1}}

\newcommand{\supp}{\operatorname{supp}} 
\newcommand{\E}{\mathbb{E}}


\title{Theory Assignment 3 -- Basic Probability, Computing and Statistics\\[2mm]
\large{Fall 2015, Master of Logic, University of Amsterdam}}

\author{}
\date{Submission deadline: Monday, September 21th, 2015, 9 a.m.}



\begin{document}
\maketitle

\paragraph{Cooperation}
Cooperation among students for both theory and programming exercises
is strongly encouraged.  However, after this discussion phase, every student writes down and submits his/her own individual solution.

\paragraph{Guidelines}
You may pick and choose {\bf 2 exercises from exercise type I}, as well as {\bf 1 from exercise types II and III each} for your submission, i.e. you need to submit {\bf a total of 4 exercises} to be able to get all points. Numbered exercises with an exclamation mark are supposed to be a bit harder and you may challenge yourself by trying to solve them.

In the directory of your private url there is folder called `theory\_submissions'. Please upload your submission there. Your submission should be a PDF-document (use a scanner for handwritten documents!) entitled \textit{AssignmentX\_yourStudentNumber.pdf}, where \textit{X} is the number of the assignment and \textit{yourStudentNumber} is your student number. If your submission does not comply with this format, we will deduct 1 point. For each day that your submission is late, we deduct 2 points. N.B.: If multiple files are submitted for a single assignment before the deadline the latest version will be graded.

If you have any question about the homework or if you need help, do not hesitate to contact \href{mailto:T.S.Brochhagen@uva.nl}{Thomas}.

\paragraph{Exercises}

\paragraph{Type I [2 exercises: 2 points per exercise]}
\begin{enumerate}
	\item Assume that 4 balls are randomly picked one after another from an urn of 20 balls, labeled from 1 to 20, without returning them. Let $X$ stand for the largest label selected. What is $P_X(X \geq 16)$?
	\item Let $X$ be a random variable taking any of the values $0$, $5$ or $-5$ with probabilities $P_X(X = -5) = .3, P_X(X = 0) = .3, P_X(X = 5) = .4$, respectively. What is $\mathbb{E}[X^2]$?
	\item Compute the probability mass function of the number of tails for 7 subsequent (fair) coin tosses. 
	\item A peddler is trying to sell her goods to two different clients. She estimates her first meeting to yield a profit with probability $.4$, and her second, independently, with probability $.7$. Any profit made is equally likely to be either from selling her luxury goods, which bring a profit of \$$1000$, or from her generic goods, for a profit of \$$500$. Compute the probability mass function of $X$, where $X$ is the total value of her profits.
	\item Let $\mathbb{E}[X] = 5$ and $var(X) = 7$, compute
		\begin{itemize}
			\item[(i)] $\mathbb{E}[(2 + X)^2]$
			\item[(ii)] $var(4 + 3X)$
		\end{itemize}
	\item[6!] Indicate the maximal number of people you can invite to your party so that the probability of any of them having the same birthday as you is less than $\frac{1}{2}$. Assume that birthdays are uniformly distributed and that we do not care about a persons birth year. 
\end{enumerate}

\paragraph{Type II [1 exercise: 3 points per exercise]}
\begin{enumerate}
\item Calculate the variance of a loaded 6-sided die that has a probability of $\frac{1}{6}$ for all odd numbered sides, and $\mathbb{P}(\{2\}) = .1,  \mathbb{P}(\{4\}) = .1, \mathbb{P}(\{6\}) = .3$.
	\item A jar contains $N$ Euro and $M$ GBP coins. Coins are taken out randomly up to the first draw of a GBP coin. If each drawn coin is put back before picking a new one, what is the probability that
		\begin{itemize}
			\item[(i)] exactly $n$ draws are needed?
			\item[(ii)] at least $k$ draws are needed?
		\end{itemize}
	\item You make a bet with a friend to the effect that he has to pay you an amount $L$ should an event $M$ happen within a year. If you estimate $M$ to happen with probability $q$ within this period, what should you charge him to enter the bet for an expected profit of $10\%$ percent of $L$?
	\item Consider a group of $n$ randomly chosen students and let $E_{i,j}$ denote the event that students $i$ and $j$ have the same birthday, $i \neq j$. Under the assumption that the students' birthdays are uniformly distributed throughout the same year, compute
		\begin{itemize}
			\item[(i)] $\mathbb{P}(E_{c,d}|E_{a,b})$
			\item[(ii)] $\mathbb{P}(E_{a,c}|E_{a,b})$
			\item[(iii)] $\mathbb{P}(E_{b,c}|E_{a,b} \cap E_{a,c})$
		\end{itemize}
	\item Let $X$ and $Y$ be two random variables with joint distribution $P_{XY}$. Show that for arbitrary 		$a,b \in \mathbb{R}$ it holds that $\E_{XY}[aX + bY] = a \E_X[X] + b \E_Y[Y].$
	\item Tall, fat people. Suppose that the average height of people in a room is 1.5m. Suppose that the average weight is 50kg.
(a) Argue that no more than one third of the population is 4.5m tall.
(b) Find an upper bound on the fraction of people who are simultanously tall (say, at least 3m) and fat (say, at least 150kg).
\end{enumerate}

	\paragraph{Type III [1 exercise: 3 points per exercise]}
\begin{enumerate}
\item For a RV $ X \sim binom(n,\theta) $ with $ n \in \mathbb{N}, n > 0 $ and $ \theta \in [0,1] $ calculate
	\begin{enumerate}
		\item the expectation $ \E[X] $ and
		\item the variance $ var(X) $.
	\end{enumerate}
	Notice that the computation should be general, i.e. neither $ n $ nor $ \theta $ should be fixed.
\item Assume that a number of independent trials, each with a probability of success of $p$, $0 < p < 1$, are performed until $q$ successes are registered. Let $X$ be equal to the number of trials required, then
	$$P(X = n) = \binom{n-1}{q-1} p^{q} (1 - p)^{n-q} \qquad \qquad n = q, q+1, ...$$

	Any RV $X$ whose probability distribution is given by the above is said to be a {\em negative binomial RV} with parameters $(q,p)$. Compute the expectation and variance of this RV with parameters $(q,p)$.
\end{enumerate}
\end{document}
